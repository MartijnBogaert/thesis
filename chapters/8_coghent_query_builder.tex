\chapter{CoGhent Data and Link Traversal}
\label{chap:coghent_link_traversal}

The primary focus of this research is the development of tools for constructing queries that target specific properties of CoGhent Human-Made Objects. These queries can either be confined to data within the CoGhent LDESs or extend beyond them by employing Link Traversal to follow links and traverse the corresponding documents. This approach facilitates the acquisition of new insights into the CoGhent data by not only enhancing the understanding of specific Human-Made Objects but also enabling their comparison in novel ways.

In the subsequent sections of this research, Comunica's link traversal capabilities will be utilized, as its modularity allows for the creation of link traversal engines tailored to the structure of the CoGhent data and the specific needs of this research. However, it is important to note that link traversal, despite its potential, remains an active area of research and can be configured in various ways.

This chapter therefore aims to explore the use of link traversal for discovering properties of Human-Made Objects, starting from the CoGhent LDESs. The chapter begins by providing an overview of the available data sources that can serve as starting points for the link traversal process. It then delves into the development of a link traversal engine optimized for the objectives outlined above. Finally, the chapter examines the most pertinent and intriguing types of resources to which the CoGhent Human-Made Objects link. These resources will be crucial for achieving the goal of broadening the knowledge of the CoGhent data.

\section{CoGhent Data Sources}
\label{sec:coghent_data_sources}

TODO

\section{Comunica Link Traversal Engine Configuration}
\label{sec:comunica_link_traversal_engine_configuration}

TODO

\section{Links to Follow}
\label{sec:links_to_follow}

TODO