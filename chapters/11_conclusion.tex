\chapter*{Conclusion}
\chaptermark{Conclusion}
\addcontentsline{toc}{chapter}{Conclusion}

The exploration of digital art collections using Link-Traversal–based Query Processing has been a multifaceted journey, intertwining the realms of art and technology. As digital art collections have become more accessible, challenges have arisen, particularly for individuals without a technical background. This research embarked on addressing these challenges, aiming to provide tools and methodologies that empower both professionals and art enthusiasts to delve deeper into the digital art landscape, making new discoveries and drawing meaningful connections.

The systematic process of discovering digital art collections can be broadly categorized into three main steps: building queries, executing them — with a specific focus on link traversal in this research — and handling the results through visualization and storage. Each chapter of this research has meticulously addressed one of these steps.

Chapter~\ref{chap:coghent_link_traversal} delved into the execution of queries using Link-Traversal-based Query Processing. The chapter emphasized the potential of link traversal in uncovering specific attributes of the Collections of Ghent's Human-Made Objects, offering insights that would remain hidden with traditional querying methods. However, it also highlighted the inherent challenges and unpredictability associated with link traversal. For instance, due to server misconfigurations, certain resources like Stad Gent's could not be accessed, while the use of others', like Getty Vocabularies', required workarounds. These challenges underscored the fragility of Link Traversal-based Query Processing compared to traditional SPARQL querying.

Chapter~\ref{chap:tools_query_building} introduced tools that simplify the intricate task of query formulation, making it more accessible to a broader audience. While these tools are valuable, they are not presented as the definitive solutions. Instead, their core query-building functionality is designed to be modular, allowing others to adapt and use it for their own discovery applications.

Chapter~\ref{chap:handling_query_results} addressed the post-query phase, focusing on the visualization and preservation of query results. The chapter weighed the advantages and disadvantages of different visualization and preservation methods. However, it also highlighted areas for further exploration, such as the adaptability of visualization tools to real-time query updates and the potential for presenting query results as immersive, interactive narratives. These areas present exciting avenues for future research.

In conclusion, this research has provided valuable insights and tools for the discovery process of digital art collections, while also shedding light on the inherent challenges of the process. Specifically, the fragility of link traversal, its slower performance compared to traditional SPARQL querying, and the unpredictability of outcomes due to server misconfigurations or other unforeseen technical issues, pose significant hurdles. However, the undeniable potential of link traversal to uncover hidden data and offer deeper insights into digital art collections offers a promising future. As technology evolves and these challenges are addressed, it is anticipated that link traversal and its associated tools will become more mainstream, benefiting a wider audience. Yet, it is important to note that, at the time of this research's publication, harnessing its full capabilities still demands a certain level of technical expertise.

\phantomsection
\section*{Ethical and social reflection}
\addcontentsline{toc}{section}{Ethical and social reflection}

TODO

%This section is required only for industrial engineering courses. The location of this section slightly differs from the order prescribed by the faculty. We recommend making this reflection as part of the conclusion because it allows you to easily reference results in your master's thesis itself.

%You can look up more information at https://www.sdgs.be/nl/sdgs
