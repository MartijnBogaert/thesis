\chapter*{Abstract}
\chaptermark{Abstract}
\addcontentsline{toc}{chapter}{Abstract}

\section*{English}

This master's thesis explores the innovative approach of discovering digital art collections using Link-Traversal-based Querying, focusing on the Collections of Ghent (CoGhent) data. CoGhent, a former partnership that digitized collections from cultural institutions, published the data as Linked Data in RDF format. By employing link traversal, this data can be explored in new ways, offering fresh insights into art collections.

The Comunica platform is central to this process, allowing for link traversal of RDF datasets and enabling the extraction of valuable data. In the CoGhent data, for instance, each entity referred to as a \textit{Human-Made Object}, such as an art piece, links to a IIIF Manifest. This manifest is a JSON-LD document that specifies artwork data and may provide instructions for digital display. Particularly, it holds a link to a picture of the piece, offering a visual representation of the artwork.

However, some resource links in the CoGhent data, notably Getty Vocabularies links, do not return an RDF compliant document, presenting a challenge for the Comunica link traversal engine. Workarounds are needed to reach the RDF compliant counterparts of these non-RDF compliant documents.

To make the discovery of CoGent collections accessible and assist art enthusiasts or professionals without a technical background in constructing SPARQL queries for a link traversal engine, two web application ideas are proposed. The first allows users to select predetermined properties of artworks, accompanied by a question indicating the purpose of the property. Each property corresponds to a sequence of predicates, which the application can ultimately use to generate a query. The second idea enables users to start from a resource of their choice, build a tree of predicates and objects, and eventually select objects of interest for query construction.

Ultimately, the discovered data can be incorporated in a IIIF Manifest, allowing display using a IIIF Viewer. This approach enhances the accessibility of art collections and provides a novel way to explore the rich cultural heritage in the CoGhent data.

\newpage
\newpagestyle{plainpage}{
  \sethead{}{}{}
  \setfoot{}{\thepage}{}
  \setheadrule{0pt}
}
\pagestyle{plainpage}

\section*{Nederlands}

Deze masterproef onderzoekt hoe digitale kunstcollecties met behulp van Link-Traversal-based Querying verkend kunnen worden. De focus ligt daarbij op de data die gepubliceerd werd door de Collectie van de Gentenaar (CoGent). Dit gewezen samenwerkingsverband digitaliseerde collecties van culturele instellingen en publiceerde de gegevens als Linked Data in RDF-formaat. Door link traversal te gebruiken, kunnen deze gegevens op nieuwe manieren worden verkend, wat leidt tot nieuwe inzichten in kunstcollecties.

Het Comunica-platform speelt een centrale rol in dit proces. Het maakt link traversal van RDF-datasets mogelijk, waardoor waardevolle gegevens kunnen worden verkregen. In de CoGent-data, bijvoorbeeld, verwijst elke entiteit die wordt aangeduid als een \textit{Mensgemaakt Object}, zoals een kunstwerk, naar een IIIF Manifest. Dit manifest is een JSON-LD-document dat kunstwerkgegevens specificeert en mogelijk instructies geeft voor digitale weergave. In het bijzonder bevat het een link naar een afbeelding van het stuk.

Echter, sommige resource-links in de CoGent-data, met name Getty Vocabularies-links, geven geen RDF-conform document terug. Hier kan Comunica niet mee aan de slag. Er zijn dus oplossingen nodig om de RDF-conforme tegenhangers te bereiken.

Om de ontdekking van CoGent-collecties toegankelijk te maken en kunstliefhebbers of professionals zonder technische achtergrond te helpen bij het opstellen van SPARQL-queries voor een link traversal engine, worden twee ideeën voor webapplicaties voorgesteld. Het eerste stelt gebruikers in staat om vooraf bepaalde eigenschappen van kunstwerken te selecteren, vergezeld van een vraag die het doel van de eigenschap aangeeft. Elke eigenschap komt overeen met een aaneenschakeling van predicaten, waarmee de applicatie uiteindelijk een query kan genereren. Het tweede idee stelt gebruikers in staat om vanuit een resource naar keuze, een boom van predicates en objecten op te bouwen, en finaal objecten van belang te selecteren voor queryconstructie.

Uiteindelijk kunnen de ontdekte gegevens worden toegewezen aan een IIIF Manifest, waarna een IIIF Viewer ze kan visualiseren. Deze benadering vergroot de toegankelijkheid van kunstcollecties en biedt een nieuwe manier om het rijke culturele erfgoed in de CoGent-data te verkennen.