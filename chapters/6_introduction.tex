\chapter*{Introduction}
\label{chap:intro}

Digital art collections have long stood as a testament to human creativity and cultural evolution. With the advent of technology, many of these collections have undergone digitization, making them more accessible to a global audience. This digitization not only preserves the integrity of the artworks but also offers an opportunity for deeper exploration and understanding. However, with this digital transformation comes a set of challenges, especially for those without a technical background. Professionals in the cultural domain and general art enthusiasts, while passionate about art, may not possess the technical expertise to navigate and query these digitized datasets. This limitation can hinder their ability to make new discoveries and truly immerse themselves in the digital art world.

Discovering art collections can be interpreted in myriad ways. At its core, discovery is about unearthing new insights, understanding the nuances of each artwork, and drawing connections that might not be immediately apparent. This research primarily focuses on retrieving the inherent properties of cultural objects, delving into the intricate details that make each piece unique. However, the true potential of discovery lies in going beyond the confines of a single dataset. Link traversal offers this opportunity, allowing for a broader exploration that extends beyond the immediate dataset, unveiling new layers of knowledge and understanding.

By employing link traversal, one can uncover hidden relationships, gain a deeper understanding of cultural objects, and even compare different artworks in novel and enlightening ways. This approach is particularly beneficial when exploring the Collections of Ghent (CoGhent), a collaborative initiative between various cultural institutions. Published in a Linked Data format, the CoGhent collections are primed for link traversal, enabling a richer and more comprehensive exploration.

This research situates itself at the intersection of art and technology, aiming to bridge the gap between the two. It seeks to empower both professionals and art enthusiasts to navigate the digital art landscape, harnessing the power of link traversal to make new discoveries and draw meaningful connections. Through a systematic exploration of the Collections of Ghent and the development of tools tailored for query formulation, this research offers a roadmap for discovering digital art collections in their entirety.

Chapter~\ref{chap:rel_work} elucidates the foundational concepts of Linked Data and their real-world applications. It delves into the core principles, data modeling, and various RDF syntaxes, setting the stage for a deeper exploration of link traversal in the subsequent chapters. 

Chapter~\ref{chap:coghent_link_traversal} focuses on the CoGhent collections, highlighting the potential of link traversal for discovering properties of Human-Made Objects. It provides an overview of the available data sources and the development of a link traversal engine optimized for the objectives of this research.

In Chapter~\ref{chap:tools_query_building}, the emphasis shifts to the development of user-centric tools for query formulation. Two conceptual web applications are introduced, designed to alleviate the technical complexities of query formulation for users. The chapter also discusses the fundamental functionality shared by both web applications, ensuring a cohesive exploration throughout.

Lastly, Chapter~\ref{chap:handling_query_results} addresses the challenges of visualizing and preserving query results. It offers an overview of potential solutions, outlining their advantages and drawbacks, ensuring that the treasures within the CoGhent collections are accessible and meaningful to all.