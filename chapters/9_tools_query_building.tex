\chapter{Tools for Query Building}
\label{chap:tools_query_building}

TODO

\section{Building Queries from Predicate Sequences}
\label{sec:building_queries_predicate_sequences}

TODO (see LateX comments for notes)


% Easiest way to rebuild existing app:
%
% - Keep dict of properties (e.g. title, description ...) and their necessary statements for WHERE clause
%   E.g.:
%   title: '?o cidoc:P102_has_title ?title.'
%   OR for properties needing more than one statement
\begin{listing}[htbp]
    \begin{minted}[samepage,fontsize=\small]{js}
objectname: [
    '?o cidoc:P41i_was_classified_by ?classified.',
    '?classified cidoc:P42_assigned ?assigned.',
    '?assigned skos:prefLabel ?objectname.',
]
    \end{minted}
    \caption{WHERE clause statements to query for \textit{objectname} stored as elements in array}
    \label{lst:where_statements_array}
\end{listing}
%   Above can also be saved as one single string
%
% - Always add all prefixes at top of query:
\begin{listing}[htbp]
    \begin{minted}[samepage,fontsize=\small]{sparql}
PREFIX cidoc:<http://www.cidoc-crm.org/cidoc-crm/>
PREFIX adms:<http://www.w3.org/ns/adms#>
PREFIX skos:<http://www.w3.org/2004/02/skos/core#>
PREFIX la:<https://linked.art/ns/terms/>
    \end{minted}
    \caption{All possible PREFIX statements of original CoGhent Query Builder}
    \label{lst:prefix_statements_original}
\end{listing}
%
% Downsides: WHERE clause statements very much hardcoded + hardcoded prefixes not good for modularity app



% Other idea:
%
% in fact, only predicates are important -> save only those
% together with predicate, save prefix
% E.g.:
\begin{listing}[htbp]
    \begin{minted}[samepage,fontsize=\small]{js}
objectname: [
    { prefix: 'cidoc', predicate: 'P41i_was_classified_by' },
    { prefix: 'cidoc', predicate: 'P42_assigned' },
    { prefix: 'skos', predicate: 'prefLabel' },
]
    \end{minted}
    \caption{Prefixes and predicates for WHERE clause statements to query for \textit{objectname} stored as elements in array}
    \label{lst:where_statements_array_predicates}
\end{listing}
%
% Downsides:
% - app needs to come up with variable names itself
%   e.g.
\begin{listing}[htbp]
    \begin{minted}[samepage,fontsize=\small]{sparql}
# Objectname
?o cidoc:P41i_was_classified_by ?s1.
?s1 cidoc:P42_assigned ?s2.
?s2 skos:prefLabel ?s3.
    \end{minted}
    \caption{WHERE clause statements with object variable names constructed using numbers}
    \label{lst:where_statements_objects_numbers}
\end{listing}
%   OR
\begin{listing}[htbp]
    \begin{minted}[samepage,fontsize=\small]{sparql}
# Objectname
?o
    cidoc:P41i_was_classified_by
        ?o_cidoc_P41i_was_classified_by.
?o_cidoc_P41i_was_classified_by
    cidoc:P42_assigned
        ?o_cidoc_P41i_was_classified_by_cidoc_P42_assigned.
?o_cidoc_P41i_was_classified_by_cidoc_P42_assigned
    skos:prefLabel
        ?o_cidoc_P41i_was_classified_by_cidoc_P42_assigned_skos_prefLabel.
    \end{minted}
    \caption{WHERE clause statements with object variable names constructed from preceding statements}
    \label{lst:where_statements_objects_long}
\end{listing}
%   -> both very unreadable
%   -> query builder not only to help build queries but also to help understand queries and data
%   -> better to have readable variables (proposed by humans up front) OR use no variables (property path sequence)
%
% Upside: keeping track of "bare" predicates one by one can help "clean up" query later on when start statements might overlap
% E.g.:
\begin{listing}[htbp]
    \begin{minted}[samepage,fontsize=\small]{sparql}
# Place
?o cidoc:P108i_was_produced_by ?produced.
?produced cidoc:P7_took_place_at ?tookplace.
?tookplace la:equivalent ?plaatsequivalent.
?plaatsequivalent skos:prefLabel ?plaats.
# Date
?produced cidoc:P4_has_time-span ?timespan.
    \end{minted}
    \caption{WHERE clause statements without overlapping statements}
    \label{lst:where_statements_no_overlapping}
\end{listing}
% instead of
\begin{listing}[htbp]
    \begin{minted}[samepage,fontsize=\small]{sparql}
# Place
?o cidoc:P108i_was_produced_by ?produced.
?produced cidoc:P7_took_place_at ?tookplace.
?tookplace la:equivalent ?plaatsequivalent.
?plaatsequivalent skos:prefLabel ?plaats.
# Date
?o cidoc:P108i_was_produced_by ?produced.
?produced cidoc:P4_has_time-span ?timespan.
    \end{minted}
    \caption{WHERE clause statements with overlapping statements}
    \label{lst:where_statements_overlapping}
\end{listing}
% Might make a difference in length of query when more statements overlap, but when the query is run, Comunica will "clean things up" for us, so doesn't have influence on data retrieval (TODO: check if true)
% To keep queries readable & app not too complicated -> let's not worry about such "feature"



% Compromise:
%
% app expects dict with properties (keys) and corresponding necessary statements (values)
% as seen before, values are arrays containing objects
% each object will be mapped to a triple (or will be part of property path sequence) in the WHERE clause, and:
% - must have predicate ('predicate')
% - may have prefix (for predicate) ('prefix')
% - may have variable name for predicate's object ('object_variable_name')
%   in case not present, construct property path sequence (no need for object variable names)
%
% dict with necessary prefixes mapped to uri's also fed to app
%
% An example:
\begin{listing}[htbp]
    \begin{minted}[samepage,fontsize=\small]{js}
const properties = {
    title: [
        { prefix: 'cidoc', predicate: 'P102_has_title' }
    ],
    description: [
        { prefix: 'cidoc', predicate: 'P3_has_note', object_variable_name: 'note' },
    ],
    objectname: [
        { prefix: 'cidoc', predicate: 'P41i_was_classified_by' },
        { prefix: 'cidoc', predicate: 'P42_assigned' },
        { prefix: 'skos', predicate: 'prefLabel' },
    ],
    association: [
        { prefix: 'cidoc', predicate: 'P128_carries' },
        { prefix: 'cidoc', predicate: 'P129_is_about', object_variable_name: 'about' },
        { prefix: 'cidoc', predicate: 'P2_has_type' },
        { prefix: 'skos', predicate: 'prefLabel' },
    ],
};

const prefixes = {
    cidoc: 'http://www.cidoc-crm.org/cidoc-crm/',
    skos: 'http://www.w3.org/2004/02/skos/core#',
};
    \end{minted}
    \caption{Properties and prefixes ready to be consumed by query building function}
    \label{lst:properties_prefixes_query_build_function}
\end{listing}
% results in (title has no object var name, description has, objectname hasn't -> property path sequence, assocation has both)
\begin{listing}[htbp]
    \begin{minted}[samepage,fontsize=\small]{sparql}
PREFIX cidoc:<http://www.cidoc-crm.org/cidoc-crm/>
PREFIX skos:<http://www.w3.org/2004/02/skos/core#>

SELECT ?title ?note ?objectname ?association

WHERE {
    # title
    ?human_made_object cidoc:P102_has_title ?title.
    
    # description
    ?human_made_object cidoc:P3_has_note ?note.
    
    # objectname
    ?human_made_object
        cidoc:P41i_was_classified_by/cidoc:P42_assigned/skos:prefLabel
            ?objectname.
    
    # association
    ?human_made_object cidoc:P128_carries/cidoc:P129_is_about ?about.
    ?about cidoc:P2_has_type/skos:prefLabel ?association.
}
    \end{minted}
    \caption{SPARQL query generated from input displayed in Code Fragment \ref{lst:properties_prefixes_query_build_function}}
    \label{lst:query_generated_edge_cases}
\end{listing}



% FILTER
% possible to create extra dict with property names as keys and filter details as values and feed to app
% but having too many dicts becomes little chaotic
% better to extend original propeties dict so it can house extra information (like filter details) apart from statements
%
% original app: only filter using string values
% this version: make multiple types of filter possible
% e.g. give string for string filtering (regex, case insensitive) + give language for language filtering
% app functionality can easily be extended for more filter types

% OPTIONAL
% extend original app functionality with optional properties
% in properties dict: 'optional' (key) - boolean (value)

% EXAMPLE with filter & optional
\begin{listing}[htbp]
    \begin{minted}[samepage,fontsize=\small]{js}
const properties = {
    title: {
        statements: [
            { prefix: 'cidoc', predicate: 'P102_has_title' }
        ],
    },
    description: {
        statements: [
            { prefix: 'cidoc', predicate: 'P3_has_note', object_variable_name: 'note' },
        ],
        filters: { string: 'luchter', language: 'nl' },
        optional: true,
    },
};
    \end{minted}
    \caption{Example of properties dictionary to illustrate use of filters and optionals}
    \label{lst:properties_filter_optionals}
\end{listing}
% results in (note filters and optional keys)
\begin{listing}[htbp]
    \begin{minted}[samepage,fontsize=\small]{sparql}
PREFIX cidoc:<http://www.cidoc-crm.org/cidoc-crm/>

SELECT ?title ?note

WHERE {
    # title
    ?human_made_object cidoc:P102_has_title ?title.
    
    # description
    OPTIONAL {
        ?human_made_object cidoc:P3_has_note ?note.
        
        FILTER(REGEX(?note, "luchter", "i"))
        FILTER(LANG(?note) = "nl")
    }
}
    \end{minted}
    \caption{SPARQL query generated from input displayed in Code Fragment \ref{lst:properties_filter_optionals}}
    \label{lst:query_generated_filters_optionals}
\end{listing}

% FROM
%const datasets = [
%	'http://stad.gent/ldes/hva',
%	'http://stad.gent/ldes/dmg',
%	'http://stad.gent/ldes/archief',
%	'http://stad.gent/ldes/stam',
%	'http://stad.gent/ldes/industriemuseum',
%];

% LIMIT and OFFSET
% according to coghent documentation "the SparQL endpoint limits results to 1.000 lines." (https://coghent.github.io/basicqueries.html), but Cominica gives more by default: 10.000 -> what's up with this (TODO: ask)
% in app, just feed to query building function as arguments

\section{A New Query Builder}
\label{sec:new_query_builder}

TODO

\section{Discovering Predicate Sequences}
\label{sec:discovering_predicate_sequences}

TODO