\pagestyle{plainpage}

\chapter*{Acknowledgements}

There are many people I would like to thank for their valuable advice and support over the past months. During those months, I have had the privilege of immersing myself in what has been an entirely new world for me, at the same time collaborating with very talented people. However, the past months have also been challenging. Therefore, a heartfelt thank you to those who have helped me navigate through them, is more than fitting.

First and foremost, I would like to express my sincere gratitude to Bryan-Elliott Tam. Bryan was my counselor throughout the second and third semesters of the academic year, going above and beyond to navigate me through the realm of link traversal. I distinctly recall one of our initial meetings. Bryan had prepared an entire PowerPoint presentation to help me get started with the new approach for my thesis, which we had decided on just the week before. I was able to dive right in. However, what I am most thankful to Bryan for, are his numerous reassuring and encouraging words during the more challenging moments. Especially during the last month, it brought me a great deal of comfort. So, Bryan, from the bottom of my heart: thank you.

Omdat ik met de volgende mensen in het dagelijkse leven steeds in mijn moedertaal communiceer, schakel ik even over naar het Nederlands. Dat doe ik in eerste instantie om Brecht Van de Vyvere te bedanken. Brecht was mijn begeleider tijdens het eerste semester van het academiejaar en heeft me mijn eerste stappen in de wereld van Linked Data helpen zetten. Dat deed hij altijd met veel zorg en de grootste glimlach. Dankjewel, Brecht.

Iemand die ik zeker niet mag en kan vergeten te bedanken, is Olivier Van D'huynslager. Als digitaal hoofd van CoGent heeft Olivier me bijna een volledig jaar lang mee begeleid. Op bijna elke meeting die ik met mijn begeleider hield, was Olivier aanwezig. Hij nam die extra vergaderingen er met de glimlach bij. Bedankt voor al je ideeën en goede raad, Olivier.

Ook Pieter Colpaert wil ik bedanken. Hij was anderhalf jaar geleden degene die me tijdens een wandeling in de Blaarmeersen kennis liet maken met de wereld van Linked Data. Dat deed hij met veel overgave en passie. Ik hoefde dan ook niet lang te twijfelen welk masterproefonderwerp ik zou kiezen. Bedankt, Pieter.

Dan zijn we aangekomen bij mijn familie. Ik ga het mezelf niet te moeilijk maken en meteen met mijn ouders beginnen. Zij zullen namelijk ook gemerkt hebben dat dit laatste jaar voor mij veruit de meest uitdagende van de voorbije zes was. Gelukkig heb ik de beste ouders die iemand zich kan wensen, en stonden zij trouw op de eerste rij om me er met aanmoedigende woorden en veel liefde doorheen te loodsen. Mama en papa, ik zal het jullie nog wel luidop zeggen, maar hier staat het alvast zwart op wit: dankjewel.

Wie ook zeker niet mag ontbreken op deze pagina, zijn mijn grootouders. Niet alleen tijdens het schrijven van mijn masterproef, maar jarenlang, tijdens elke examenperiode, mocht ik bij hen - gewoon het hoekje om - in alle rust komen werken. Elke dag opnieuw werd ik beladen met lekker eten, tussendoortjes, aanmoedigingen en vooral veel liefde. Ik heb veel onvergetelijke momenten meegemaakt tijdens mijn studententijd, maar ik lieg niet als ik zeg dat ik de dagen bij hen nog het meest zal missen. Lieve moemoe en grootva, ik ben jullie eeuwig dankbaar.

En dan is er nog iemand die ongetwijfeld niet kan wachten haar naam te horen. Mijn allerliefste Eva, dankjewel voor al die weken, maanden en jaren onophoudelijke steun. Ik weet niet hoe je het doet, maar zelfs op de moeilijkste momenten slaag je er telkens weer in mij op te rapen en met nieuwe moed vooruit te doen kijken. Ik kan niet wachten om met jou de volgende fase van mijn - ons - leven te beginnen. Ik zie je graag.

Ten slotte wil ik ook alle mensen die ik nog niet vermeld heb maar die me toch al die tijd gesteund hebben, heel oprecht bedanken. Ik denk daarbij aan familie, vrienden, medestudenten ... Die steun hoeft zelfs niet altijd uitgesproken te zijn. Een schouderklopje of een aanmoedigende glimlach kan al een wereld van verschil maken. Het zijn de kleine gebaren die het 'm doen. Dankjewel allemaal!