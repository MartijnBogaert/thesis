\appendix
\renewcommand{\thesection}{\Alph{section}}
\setcounter{section}{0}

\begin{appendices}

\section{Notes on the usage of AI}
\label{app:ai}

In accordance with the guidelines\footnote{\url{https://masterproef.tiwi.ugent.be/scriptie/inhoud/Richtlijn\%20AI\%20gebruik\%20in\%20masterproef.pdf}} on the master's thesis for students pursuing the \textit{Master of Science in Industriële Wetenschappen: informatica} at Ghent University, this appendix provides a detailed account of the AI tools utilized during the research and composition of this thesis.

Primarily, it is crucial to stress that the research itself was solely conducted through human endeavor. However, during the research process, AI tools, particularly OpenAI's \textit{ChatGPT}, were occasionally consulted for assistance. Specifically, the capabilities of the \textit{GPT-3.5} LLM were instrumental in two specific areas of the thesis development.

Firstly, ChatGPT was frequently consulted during the development of the applications associated with this research. This involved the kind of questions that developers typically consult forums like \textit{Stack Overflow} for. The advantage of directing such questions to ChatGPT, however, lies in its ability to tailor responses based on the specific context provided, offering more personalized solutions.

Secondly, ChatGPT was invaluable during the thesis writing phase. After all, given that GPT-3.5 is an LLM, it excels in producing coherent and fluent texts. However, it is imperative to clarify that the tool was never used to generate original content or insights. Such an approach would not only exceed the capabilities of GPT 3.5 but also and most importantly violate academic integrity. Instead, ChatGPT was employed for tasks like \textit{translating the given sentence(s}) or \textit{rewrite the following sentence(s) in smoother English}. This facilitated the writing process for a non-native English speaker and ensured the final text to be more comprehensible.

\newpage
\section{RDF Syntaxes}

TODO

\end{appendices}
